\documentclass{scrartcl}
\usepackage{fontspec}
\usepackage{polyglossia}
\usepackage[compact]{titlesec}

\setmainlanguage{french}
\title{Manuel d'utilisation du programme de création d'emploi du temps
\emph{Crispy Chainsaw}}
\author{}
\date{}

\begin{document}
\maketitle

\section{Éléments fournis}
\begin{itemize}
  \item Exécutable du programme \texttt{crispy\_chainsaw.exe},
  \item base de donnée pré remplie \texttt{dummy.db}.
\end{itemize}

\section{Installation}
Placer le fichier exécutable dans le même dossier que le fichier de base de
donnée.

\section{Hypothèses de travail}
\begin{itemize}
  \item Les B737 effectuent 3 vols par jour,
  \item les B727 effectuent 2 vols par jour.
\end{itemize}

\section{Onglet pilotes}
Le premier onglet est dédié à la gestion du personnel. Un membre du personnel
peut être sélectionné en cliquant sur son identifiant dans la colonne de gauche.
Une fois la personne sélectionnée, il est possible
\begin{itemize}
  \item de modifier, supprimer un membre du personnel,
  \item d'imposer un jour de travail à un personnel,
  \item récapitulatif du planning par personne entre deux dates choisies, en
    ayant au préalable appuyé sur rafraîchir.
\end{itemize}

\section{Onglets appareils}
Chaque onglet appareil permet de présenter l'emploi du temps qui y est associe.
La partie supérieure de l'onglet présente deux boutons en plus des choix de
date. Ceux ci permettent de
\begin{itemize}
  \item \emph{Valider}: recalculer les emplois du temps,
  \item \emph{Modifier durée du vol}: spécifier la durée d'un vol effectue.
\end{itemize}

\paragraph{Édition de la durée}
L'édition de la durée de vol entraine l'ouverture d'une fenêtre qui permet
\begin{itemize}
  \item d'annuler le vol,
  \item de spécifier la durée d'un vol, si celui là a été effectué.
\end{itemize}

\section{Menu d'export}
Le menu export du bandeau supérieur permet d'exporter un emploi du temps ou les
bilans des pilotes au
format \texttt{csv}. Ces fichiers peuvent ensuite être interprétés par e.g.\ des
tableurs.

\section{Calcul des emplois du temps}
L'algorithme assigne automatiquement les vols aux personnes de la manière
suivante
\begin{enumerate}
  \item si le statut  a été spécifié via l'onglet pilotes, alors l'emploi du
    temps ne le modifie pas,
  \item sinon, tous les pilotes disponibles sont récupérés et un si un membre du
    personnel peut occuper la fonction, alors la fonction lui est attribuée,
  \item on passe au vol suivant.
\end{enumerate}

\end{document}
